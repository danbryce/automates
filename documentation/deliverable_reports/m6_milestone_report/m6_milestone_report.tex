% !TEX TS-program = lualatex
% !TEX encoding = UTF-8 Unicode
% !BIB program = biber 

\documentclass[article, 12pt, oneside]{memoir}
\usepackage[dvipsnames]{xcolor}
\usepackage{amsmath, amssymb}
\usepackage{xspace}
\usepackage{hyperref}
\usepackage{longtable}
\usepackage{fancyvrb}
\usepackage{listings}
\usepackage{color}
\usepackage{fancyvrb}
\newcommand{\VerbBar}{|}
\newcommand{\VERB}{\Verb[commandchars=\\\{\}]}
\DefineVerbatimEnvironment{Highlighting}{Verbatim}{commandchars=\\\{\}}
% Add ',fontsize=\small' for more characters per line
\newenvironment{Shaded}{}{}
\newcommand{\AlertTok}[1]{\textcolor[rgb]{1.00,0.00,0.00}{\textbf{#1}}}
\newcommand{\AnnotationTok}[1]{\textcolor[rgb]{0.38,0.63,0.69}{\textbf{\textit{#1}}}}
\newcommand{\AttributeTok}[1]{\textcolor[rgb]{0.49,0.56,0.16}{#1}}
\newcommand{\BaseNTok}[1]{\textcolor[rgb]{0.25,0.63,0.44}{#1}}
\newcommand{\BuiltInTok}[1]{#1}
\newcommand{\CharTok}[1]{\textcolor[rgb]{0.25,0.44,0.63}{#1}}
\newcommand{\CommentTok}[1]{\textcolor[rgb]{0.38,0.63,0.69}{\textit{#1}}}
\newcommand{\CommentVarTok}[1]{\textcolor[rgb]{0.38,0.63,0.69}{\textbf{\textit{#1}}}}
\newcommand{\ConstantTok}[1]{\textcolor[rgb]{0.53,0.00,0.00}{#1}}
\newcommand{\ControlFlowTok}[1]{\textcolor[rgb]{0.00,0.44,0.13}{\textbf{#1}}}
\newcommand{\DataTypeTok}[1]{\textcolor[rgb]{0.56,0.13,0.00}{#1}}
\newcommand{\DecValTok}[1]{\textcolor[rgb]{0.25,0.63,0.44}{#1}}
\newcommand{\DocumentationTok}[1]{\textcolor[rgb]{0.73,0.13,0.13}{\textit{#1}}}
\newcommand{\ErrorTok}[1]{\textcolor[rgb]{1.00,0.00,0.00}{\textbf{#1}}}
\newcommand{\ExtensionTok}[1]{#1}
\newcommand{\FloatTok}[1]{\textcolor[rgb]{0.25,0.63,0.44}{#1}}
\newcommand{\FunctionTok}[1]{\textcolor[rgb]{0.02,0.16,0.49}{#1}}
\newcommand{\ImportTok}[1]{#1}
\newcommand{\InformationTok}[1]{\textcolor[rgb]{0.38,0.63,0.69}{\textbf{\textit{#1}}}}
\newcommand{\KeywordTok}[1]{\textcolor[rgb]{0.00,0.44,0.13}{\textbf{#1}}}
\newcommand{\NormalTok}[1]{#1}
\newcommand{\OperatorTok}[1]{\textcolor[rgb]{0.40,0.40,0.40}{#1}}
\newcommand{\OtherTok}[1]{\textcolor[rgb]{0.00,0.44,0.13}{#1}}
\newcommand{\PreprocessorTok}[1]{\textcolor[rgb]{0.74,0.48,0.00}{#1}}
\newcommand{\RegionMarkerTok}[1]{#1}
\newcommand{\SpecialCharTok}[1]{\textcolor[rgb]{0.25,0.44,0.63}{#1}}
\newcommand{\SpecialStringTok}[1]{\textcolor[rgb]{0.73,0.40,0.53}{#1}}
\newcommand{\StringTok}[1]{\textcolor[rgb]{0.25,0.44,0.63}{#1}}
\newcommand{\VariableTok}[1]{\textcolor[rgb]{0.10,0.09,0.49}{#1}}
\newcommand{\VerbatimStringTok}[1]{\textcolor[rgb]{0.25,0.44,0.63}{#1}}
\newcommand{\WarningTok}[1]{\textcolor[rgb]{0.38,0.63,0.69}{\textbf{\textit{#1}}}}
\newcommand{\passthrough}[1]{\lstset{mathescape=false}#1\lstset{mathescape=true}}
\usepackage[tracking, protrusion]{microtype}

\usepackage{fontspec}
\setmainfont{Alegreya}
\setsansfont{Alegreya Sans}
\setmonofont[Scale=MatchLowercase]{Menlo}
\linespread{1.1}

\hypersetup{
  unicode=true,
  linktocpage=true,
  linkcolor=NavyBlue,
  colorlinks=true,
  urlcolor=NavyBlue,
  citecolor=RoyalBlue,
  breaklinks=true,
}

\def\sectionautorefname{§}
\def\subsectionautorefname{§}

\renewcommand{\bf}{\bfseries}
\renewcommand{\it}{\itshape}

% ======================================================================
% Graphics
% ======================================================================
% ======================================================================
% Memoir layout and formatting specifications
% ======================================================================
%\setlrmarginsandblock{1.91in}{*}{*}
%\setulmarginsandblock{1in}{*}{*}
\setsecnumdepth{subsection}
\settocdepth{subsection}
\setsecheadstyle{\color{Maroon}\large\sffamily\bfseries}
\setsubsecheadstyle{\bfseries}
\setparaheadstyle{\bfseries\sffamily}
\tightlists

\counterwithout{section}{chapter}
\captionstyle[\centering]{\sffamily\small}
\captionnamefont{\color{Maroon}\sffamily\small}
\captionwidth{0.95\textwidth}
\changecaptionwidth


\checkandfixthelayout


\title{Month 6 Milestone Report}
\author{\href{https://ml4ai.github.io/automates/team/}{The AutoMATES Team}}
\date{2019-04-31}

\begin{document}
\maketitle
\tableofcontents*

\bigskip
\bigskip

\noindent \emph{Note: This PDF has been automatically generated from a web
  version, available here:}

  {
  \small
\noindent
\url{https://ml4ai.github.io/automates/documentation/deliverable_reports/m6_milestone_report}
}
\emph{Please visit the web version for the best experience.}

Link to the PDF version of this report. (TODO - insert link to PDF)

\hypertarget{overview}{%
\section{Overview}\label{overview}}

This report summarizes progress towards AutoMATES milestones at the six
month mark, emphasizing changes since the m5 report.

\hypertarget{codeexplorer-webapp}{%
\section{CodeExplorer Webapp}\label{codeexplorer-webapp}}

The webapp has been Dockerized and is available from Dockerhub. To pull
the image and run the webapp locally, do the following commands:

\begin{verbatim}
docker pull adarshp/codex
docker run -p 4000:80 adarshp/codex:latest
\end{verbatim}

Then navigate to http://127.0.0.1:4000 in your web browser.

\hypertarget{program-analysis}{%
\section{Program Analysis}\label{program-analysis}}

\hypertarget{refactoring-of-front-end}{%
\subsection{Refactoring of front-end:}\label{refactoring-of-front-end}}

The focus of work in the past month was on refactoring the ``Fortran
code -\textgreater{} XML AST -\textgreater{} Python Code'' generation
process. The refactoring was done in order address issues with the Open
Fortran Parser (OFP) AST XML generation that impacted aspects of
\texttt{translate.py} and \texttt{pyTranslate.py} (inefficiency and
inconsistencies). The new program \texttt{rectify.py} fixes these
issues, relieving \texttt{translate.py} from having to handle variations
in OFP-derived AST representaiton, and focus on generating the pickle
file for \texttt{pyTranslate.py}. \texttt{rectify.py} reduces size of
the generated AST XML by 30\% to 40\% comared to the original XML. For
example, the original \texttt{PETASCE\_markup.xml} is 4,177 lines
whereas the new XML is only 2,858 lines. This also speeds up AST
processing, reducing the number of special cases required.

\hypertarget{enhancement-of-grfn-specifications-and-added-support-for-new-fortran-language-features-in-grfn}{%
\subsection{Enhancement of GrFN Specifications and added support for new
FORTRAN language features in
GrFN}\label{enhancement-of-grfn-specifications-and-added-support-for-new-fortran-language-features-in-grfn}}

\begin{itemize}
\tightlist
\item
  Continued progress in generating GrFN specification and associated
  lambda files; this includes abstracting the multi-module structure of
  the original FORTRAN code into a multi-file GrFN JSON data structure.
  This phase we developed a strategy for representing modules and this
  is being implemented in the GrFN spec.
\item
  The Python-to-GrFN converter now handles multi-module files more
  elegantly with a separate GrFN and lambda file for each FORTRAN
  module.
\item
  Began refactoring and cleanup of the code-base of the Python-to-GrFN
  converter. This will allow for more seamless integration of newer
  FORTRAN language features to the GrFN JSON and lambda file.
\item
  Laid groundwork for the automatic support for inline comments. This
  will not necessarily be integrated in the GrFN file and might be
  provided through other means as required.
\item
  Started integration of \href{https://pytorch.org/}{PyTorch} with the
  automatically generated lambda files. This integration permits much
  faster code execution.
\item
  Continued discussion and brainstorming about how arrays and
  multidimensional data structures can be represented in GrFN among
  other features such as \texttt{DELETE}, \texttt{SAVE} and
  \texttt{GOTO}.
\end{itemize}

\hypertarget{text-reading}{%
\section{Text Reading}\label{text-reading}}

The team has been working on the following tasks (all in progress):

\begin{enumerate}
\def\labelenumi{\arabic{enumi}.}
\tightlist
\item
  Analyzing the relevant scientific publications with two goals in mind:

  \begin{itemize}
  \tightlist
  \item
    finding additional relations to include in the taxonomy, e.g.,
    calculations (``E0 is calculated as the product of Kcd and ETpm.'')
    and non-precise parameter settings (``Rns and Rnl are generally
    positive or zero in value.''); this work will help increase
    extraction coverage and will be done in close collaboration with the
    Model Analysis team to make sure the relations included in the
    taxonomy are relevant to the project needs;
  \item
    finding additional test cases to include in the test set for the
    existing relations (including both positive and negative examples);
    this work will help increase precision and recall.
  \end{itemize}
\item
  Working on the rules used for extraction: analyzing the output of the
  system for false positives and adding constraints on rules and actions
  used for extraction to eliminate the false positives; this work will
  help increase precision; writing additional rules; this work will help
  increase recall;
\item
  Implementing the functionality to extract units; this work is needed
  to substitute the
  \href{https://github.com/kermitt2/grobid}{grobid}-quantities unit
  extractor, which has shown to be inadequate for our needs---with
  grobid-quantities, units are only extracted when preceded by a value;
  the units we need to extract are frequently compound (multi-word) and
  are not consistently extracted by grobid-quantities (e.g., ``kg ha-1
  mm-1''). To this end we are including an entity finder based on a
  gazetteer in the current extraction system; Building a gazetteer (a
  lexicon) of units based on the Guide for the Use of the International
  System of Units (SI) (Thompson and Taylor, 2008) to be used with the
  gazetteer entity finder.
\end{enumerate}

In the following weeks, the team will continue to work on the current
tasks. The work on having the tests pass has been put on hold
temporarily to make sure the tests reflect the needs of the downstream
consumers (e.g., the Model Analysis team). After a meeting with the
Model Analysis team, the potential test cases will be added to the
current set of tests and the team will continue the work on having the
tests pass.

\hypertarget{equation-reading}{%
\section{Equation Reading}\label{equation-reading}}

The UA team began the task of equation reading using the open-source
\texttt{im2markup} model (Deng, Kanervisto \& Rush, 2016). However,
while the pre-trained model performed well on the authors' data, when
used on data from other sources, even after using the same preprocessing
pipeline, the decoded equations were not usable (please see detailed
results in the month 5 report). Additionally, because of the way the
code was written, any model trained on a GPU (which is necessary due to
the complexity of the model and the amount of training) would require a
GPU for inference as well. Finally, when trying to re-train the model on
our data, version conflicts with library dependencies are continually
causing the model to crash.

For these reasons, and also because we ultimately want to extend the
model, the UA team has reimplemented the model in
\href{https://pytorch.org/}{PyTorch}, which allows for trained models to
be used on either a CPU or a GPU. Additionally, the reimplementation was
intentionally done in a modular way, such that components can be
replaced/extended easily. The team plans to make use of this modularity
in the near future to add online data augmentation (to reduce
sensitivity to exact input format) and a final layer to the equation
decoder which will choose the globally optimal equation, rather than
greedily making decoding decisions at each time step.

The UA team is in the process of building a
\href{https://singularityhub.com/}{Singularity} container to train the
model on the
\href{https://it.arizona.edu/service/high-performance-computing}{UA
HPC}. Once that is completed, the team will ensure that their
reimplementation can approximately reproduce the results of the
original, and will then begin working on the needed extensions to
improve model performance for our use case.

\hypertarget{model-analysis}{%
\section{Model Analysis}\label{model-analysis}}

\hypertarget{variable-domain-constraint-propagation}{%
\subsection{Variable Domain Constraint
Propagation}\label{variable-domain-constraint-propagation}}

During phase 1 the model analysis team identified instances of input
sets to the PETASCE Evapo-transpiration model that caused bound errors
during evaluation. This is due to the bound constraints of certain
mathematical functions used in PETASCE, such as \texttt{arccos} and
\texttt{log}. These constraints can be buried deep within the model
architecture and commonly place shared constraints on more than one
variable value. Some examples of constraints found in the PETASCE model
are shared below.

\begin{Shaded}
\begin{Highlighting}[]
\NormalTok{RHMIN }\KeywordTok{=} \BuiltInTok{MAX}\NormalTok{(}\FloatTok{20.0}\NormalTok{, }\BuiltInTok{MIN}\NormalTok{(}\FloatTok{80.0}\NormalTok{, EA}\KeywordTok{/}\NormalTok{EMAX}\KeywordTok{*}\FloatTok{100.0}\NormalTok{))   }\CommentTok{! EMAX must not be 0}
\NormalTok{RNL }\KeywordTok{=} \FloatTok{4.901E-9}\KeywordTok{*}\NormalTok{FCD}\KeywordTok{*}\NormalTok{(}\FloatTok{0.34}\KeywordTok{-}\FloatTok{0.14}\KeywordTok{*}\BuiltInTok{SQRT}\NormalTok{(EA))}\KeywordTok{*}\NormalTok{TK4   }\CommentTok{! EA must be non-negative}

\CommentTok{! The inner expression is bounded by [-1, 1] and this bound propagates to}
\CommentTok{! expression of three variables.}
\NormalTok{WS }\KeywordTok{=} \BuiltInTok{ACOS}\NormalTok{(}\KeywordTok{-}\FloatTok{1.0}\KeywordTok{*}\BuiltInTok{TAN}\NormalTok{(XLAT}\KeywordTok{*}\NormalTok{PIE}\KeywordTok{/}\FloatTok{180.0}\NormalTok{)}\KeywordTok{*}\BuiltInTok{TAN}\NormalTok{(LDELTA))}
\end{Highlighting}
\end{Shaded}

Automating the process of sensitivity analysis for any input set will
require static analysis to determine the set of domain constraints that
can be used to validate an input. Then when running a model the first
step will be to validate the input set to ensure that the inputs fit
within the domains. For sensitivity analysis that includes a sampling
step over a set of bounds this will require validation of the runtime
bounds and possibly amending the sampling process for different
sensitivity methods to ensure that samples are not taken from areas out
of the constrained domain. Currently the MA team is working on a method
to determine the bounds for a model by doing a forward/backward pass of
bounds over the model, starting with calculating the range value
interval constraints of variables in a forward pass and then using this
information to calculate the variable domain value interval constraints
of variables during the backward pass.

\hypertarget{senstivity-analysis-of-petpt-priestley-taylor-model}{%
\subsection{Senstivity Analysis of PETPT (Priestley-Taylor)
model}\label{senstivity-analysis-of-petpt-priestley-taylor-model}}

Model analysis currently uses the computation of Sobol indices to
measure the variance in the output as a function of the fluctuations in
the input parameters.

The AutoMATES MA pipeline current uses the
\href{https://salib.readthedocs.io/en/latest/}{SALib} python library,
which provides functionality for three different sensitivity methods,
Sobol, Fourier Amplitude Sensitivity Test (FAST), and Random Balance
Design-Fourier Amplitude Sensitivity Test (RBD-FAST). This phase we
explored using these models for sensitivity analysis. We compared the
first order indices (S\(_i\); \(i\) is the variable) across the PETPT
and ASCE models.

(The Priestley-Taylor (PETPT) model is an example of an
evapotranspiration (EO) model where the daily EO rate of a crop depends
on the following independent variables - the maximum (TMAX) and minimum
(TMIN) observable temperatures within a given day, daily solar radiation
(SRAD), leaf area index (XHLAI), and soil albedo coefficient (MSALB).)

Our results indicate that for reasonable approximations of the input
parameter bounds, \(S_i\) values are maximum for TMAX and minimum (zero)
for XHLAI in all three methods. SRAD followed by MSALB have the most
significant \(S_i\)'s after TMAX while the sobol index for TMIN in the
PETPT model is negligible. The Sobol method can compute the second order
index (\(S_{ij}\); for the variables \(i\) and \(j\)) as well.
Interestingly, we find that while \(S_{XHLAI}\) is vanishingly small,
the second order index of XHLAI and TMAX is significantly large. Again,
\(S_{ij}\) for \((i, j)=\) (TMAX, SRAD) is non-negligible. In addition,
the runtime for each of the sensitivity analysis methods is computed and
compared for different sample sizes. It becomes evident that for large
sample sizes (\(\sim\) \(10^6\)), Sobol is two orders of magnitude
slower than both FAST and RBD-FAST methods. In summary, TMAX is the most
significant parameter and any significant change in its value will have
the maximum effect on the EO rate. Our future goal is to extend our
discussion of sensitivity analysis to the PETASCE model in which the EO
rate depends on the independent variables discussed above as well as a
new set of input parameters.

\hypertarget{code-summarization}{%
\section{Code Summarization}\label{code-summarization}}

The code summarization team has begun focusing our attention on
templated generation for summarizing the computations and structure of
our extracted scientific models from GrFN. To meet this end we have
begun investigating an architecture proposed in recent NLP literature
that deals with (Wiseman, Shieber \& Rush, 2018). The idea behind using
a templated generation method is that the source code we wish to
summarize has well-specified control-flow structures that can be easily
summarized via rules. Such structures as conditional statements, loops,
indexed loops, and function calls are all well-defined components of the
GrFN structure that allow for the construction of a skeletal summary.
The neural generation methods can then be used to create a brief
description of the actual computations being carried out in the function
nodes that are contained within a GrFN structure. We are currently in
the process of investigating what rule-based methods would be best for
summarizing whole models or portions of models. Once we have developed
templates from these rule-based methods we will begin experimenting with
neural generation methods to fill function node computation descriptions
within the templates.

\end{document}
