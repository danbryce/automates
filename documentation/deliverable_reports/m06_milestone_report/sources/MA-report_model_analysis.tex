\documentclass{article}

\begin{document}

\section{Senstivity Analysis of PETPT (Priestley-Taylor) model}

Sensitivity analysis of any theoretical model involves the computation of Sobol indices which measures the variance in the output as a
result of the fluctuations in the input parameters. 
%has been performed for the
The Priestley-Taylor (PETPT) model is an example of an
evapotranspiration (EO) model where the daily EO rate of a crop depends on the following independent
variables - the maximum (TMAX) and minimum (TMIN) observable temperatures within a
given day, daily solar radiation (SRAD), leaf area index (XHLAI), and soil
albedo coefficient (MSALB). For computational purposes, 
the SALib library in python allows one to perform a
sensitivity analysis for a given model. In our work, we have implemented three
different sensitivity methods, namely, Sobol, Fourier Amplitude Sensitivity Test or
FAST, and finally, Random Balance Design-Fourier Amplitude Sensitivity Test or
RBD-FAST and compared the first order indices (S$_i$; $i$ is the variable) across the different models.
Our results indicate that for reasonable approximations of the input parameter
bounds, S$_i$ values are maximum for TMAX and minimum (zero) for
XHLAI in all three methods. SRAD followed by MSALB have the most significant
S$_i$'s after TMAX while the sobol index for TMIN in the PETPT model is
negligible. The Sobol method can compute the second order index (S$_{ij}$; for
the variables $i$ and $j$) as
well. Interestingly, we find that while S$_{XHLAI}$ is vanishingly small,
the second order index of XHLAI and TMAX is significantly large. Again,
S$_{ij}$ for $(i, j)=$ (TMAX, SRAD) is non-negligible. In addition, the runtime for each of
the sensitivity analysis methods is computed and compared for different sample
sizes. It becomes evident that for large sample sizes ($\sim$ 10$^6$), Sobol is
two orders of magnitude slower than both FAST and RBD-FAST methods. In summary,
TMAX is the most significant parameter and any significant change in its value will
have the maximum effect on the EO rate. Our future goal is to extend our
discussion of sensitivity analysis to the PETASCE model in which the EO rate
depends on the
independent variables discussed above as well as a new set of input parameters.

\section{Parameter Estimation}

A necessary question that arises from the previous section is how reliable are
the bounds of our input variables that have been used in estimating the Sobol
indices. While such an
information can be extracted after having parsed through hundreds of documents on
various evapotranspiration models and other experimental data, we would like to
adopt a more fundamental approach to determine the domain of a high
dimensional space spanned by our input
parameters. Given our previous choice of the bounds imposed on the input parameters, it is easy
to see that there exist a subset of parameter sets which yield similar
evapotranspiration rates in both PETPT and PETASCE models. However, such an
{\it ad-hoc} choice generates only a slice of a larger domain of possible values in
the configuration space of our parameters. A more rigorous approach then
requires us to minimize the square of the difference between the EO rates of the PETPT and
PETASCE models which is defined as the Loss. 
Typically, the Loss is a non-convex, non-linear, non-separable function with
multiple minima and hence any optimizaiton procedure is non-trivial. 
We invoke the standard gradient descent algorithm that is ideal for
convex optimzation problems and apply it over here. Moreover, the mathematical
expression for the EO rates vary based on the domains of the input variables.
Hence, domain constraints need to be imposed while minimizing the Loss.
This work is currently  ongoing and we hope to develop a more efficient
algorithm that can precisely identify a list of parameter sets that optimize
the output in both models. 


\end{document}
